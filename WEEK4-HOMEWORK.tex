% Options for packages loaded elsewhere
\PassOptionsToPackage{unicode}{hyperref}
\PassOptionsToPackage{hyphens}{url}
%
\documentclass[
]{article}
\usepackage{amsmath,amssymb}
\usepackage{iftex}
\ifPDFTeX
  \usepackage[T1]{fontenc}
  \usepackage[utf8]{inputenc}
  \usepackage{textcomp} % provide euro and other symbols
\else % if luatex or xetex
  \usepackage{unicode-math} % this also loads fontspec
  \defaultfontfeatures{Scale=MatchLowercase}
  \defaultfontfeatures[\rmfamily]{Ligatures=TeX,Scale=1}
\fi
\usepackage{lmodern}
\ifPDFTeX\else
  % xetex/luatex font selection
\fi
% Use upquote if available, for straight quotes in verbatim environments
\IfFileExists{upquote.sty}{\usepackage{upquote}}{}
\IfFileExists{microtype.sty}{% use microtype if available
  \usepackage[]{microtype}
  \UseMicrotypeSet[protrusion]{basicmath} % disable protrusion for tt fonts
}{}
\makeatletter
\@ifundefined{KOMAClassName}{% if non-KOMA class
  \IfFileExists{parskip.sty}{%
    \usepackage{parskip}
  }{% else
    \setlength{\parindent}{0pt}
    \setlength{\parskip}{6pt plus 2pt minus 1pt}}
}{% if KOMA class
  \KOMAoptions{parskip=half}}
\makeatother
\usepackage{xcolor}
\usepackage[margin=1in]{geometry}
\usepackage{color}
\usepackage{fancyvrb}
\newcommand{\VerbBar}{|}
\newcommand{\VERB}{\Verb[commandchars=\\\{\}]}
\DefineVerbatimEnvironment{Highlighting}{Verbatim}{commandchars=\\\{\}}
% Add ',fontsize=\small' for more characters per line
\usepackage{framed}
\definecolor{shadecolor}{RGB}{248,248,248}
\newenvironment{Shaded}{\begin{snugshade}}{\end{snugshade}}
\newcommand{\AlertTok}[1]{\textcolor[rgb]{0.94,0.16,0.16}{#1}}
\newcommand{\AnnotationTok}[1]{\textcolor[rgb]{0.56,0.35,0.01}{\textbf{\textit{#1}}}}
\newcommand{\AttributeTok}[1]{\textcolor[rgb]{0.13,0.29,0.53}{#1}}
\newcommand{\BaseNTok}[1]{\textcolor[rgb]{0.00,0.00,0.81}{#1}}
\newcommand{\BuiltInTok}[1]{#1}
\newcommand{\CharTok}[1]{\textcolor[rgb]{0.31,0.60,0.02}{#1}}
\newcommand{\CommentTok}[1]{\textcolor[rgb]{0.56,0.35,0.01}{\textit{#1}}}
\newcommand{\CommentVarTok}[1]{\textcolor[rgb]{0.56,0.35,0.01}{\textbf{\textit{#1}}}}
\newcommand{\ConstantTok}[1]{\textcolor[rgb]{0.56,0.35,0.01}{#1}}
\newcommand{\ControlFlowTok}[1]{\textcolor[rgb]{0.13,0.29,0.53}{\textbf{#1}}}
\newcommand{\DataTypeTok}[1]{\textcolor[rgb]{0.13,0.29,0.53}{#1}}
\newcommand{\DecValTok}[1]{\textcolor[rgb]{0.00,0.00,0.81}{#1}}
\newcommand{\DocumentationTok}[1]{\textcolor[rgb]{0.56,0.35,0.01}{\textbf{\textit{#1}}}}
\newcommand{\ErrorTok}[1]{\textcolor[rgb]{0.64,0.00,0.00}{\textbf{#1}}}
\newcommand{\ExtensionTok}[1]{#1}
\newcommand{\FloatTok}[1]{\textcolor[rgb]{0.00,0.00,0.81}{#1}}
\newcommand{\FunctionTok}[1]{\textcolor[rgb]{0.13,0.29,0.53}{\textbf{#1}}}
\newcommand{\ImportTok}[1]{#1}
\newcommand{\InformationTok}[1]{\textcolor[rgb]{0.56,0.35,0.01}{\textbf{\textit{#1}}}}
\newcommand{\KeywordTok}[1]{\textcolor[rgb]{0.13,0.29,0.53}{\textbf{#1}}}
\newcommand{\NormalTok}[1]{#1}
\newcommand{\OperatorTok}[1]{\textcolor[rgb]{0.81,0.36,0.00}{\textbf{#1}}}
\newcommand{\OtherTok}[1]{\textcolor[rgb]{0.56,0.35,0.01}{#1}}
\newcommand{\PreprocessorTok}[1]{\textcolor[rgb]{0.56,0.35,0.01}{\textit{#1}}}
\newcommand{\RegionMarkerTok}[1]{#1}
\newcommand{\SpecialCharTok}[1]{\textcolor[rgb]{0.81,0.36,0.00}{\textbf{#1}}}
\newcommand{\SpecialStringTok}[1]{\textcolor[rgb]{0.31,0.60,0.02}{#1}}
\newcommand{\StringTok}[1]{\textcolor[rgb]{0.31,0.60,0.02}{#1}}
\newcommand{\VariableTok}[1]{\textcolor[rgb]{0.00,0.00,0.00}{#1}}
\newcommand{\VerbatimStringTok}[1]{\textcolor[rgb]{0.31,0.60,0.02}{#1}}
\newcommand{\WarningTok}[1]{\textcolor[rgb]{0.56,0.35,0.01}{\textbf{\textit{#1}}}}
\usepackage{graphicx}
\makeatletter
\def\maxwidth{\ifdim\Gin@nat@width>\linewidth\linewidth\else\Gin@nat@width\fi}
\def\maxheight{\ifdim\Gin@nat@height>\textheight\textheight\else\Gin@nat@height\fi}
\makeatother
% Scale images if necessary, so that they will not overflow the page
% margins by default, and it is still possible to overwrite the defaults
% using explicit options in \includegraphics[width, height, ...]{}
\setkeys{Gin}{width=\maxwidth,height=\maxheight,keepaspectratio}
% Set default figure placement to htbp
\makeatletter
\def\fps@figure{htbp}
\makeatother
\setlength{\emergencystretch}{3em} % prevent overfull lines
\providecommand{\tightlist}{%
  \setlength{\itemsep}{0pt}\setlength{\parskip}{0pt}}
\setcounter{secnumdepth}{-\maxdimen} % remove section numbering
\ifLuaTeX
  \usepackage{selnolig}  % disable illegal ligatures
\fi
\usepackage{bookmark}
\IfFileExists{xurl.sty}{\usepackage{xurl}}{} % add URL line breaks if available
\urlstyle{same}
\hypersetup{
  pdftitle={CASA0003\_ Week4\_Homework},
  pdfauthor={Fiorella Guillen Hurtado},
  hidelinks,
  pdfcreator={LaTeX via pandoc}}

\title{CASA0003\_ Week4\_Homework}
\author{Fiorella Guillen Hurtado}
\date{2024-10-28}

\begin{document}
\maketitle

\subsection{Comparing global gender inequality index between 2019 and
2010}\label{comparing-global-gender-inequality-index-between-2019-and-2010}

\paragraph{Load all libraries}\label{load-all-libraries}

Loading libraries for data reading, data wrangling and plotting.

\begin{Shaded}
\begin{Highlighting}[]
\FunctionTok{library}\NormalTok{(here)}
\FunctionTok{library}\NormalTok{(sf)}
\FunctionTok{library}\NormalTok{(tidyverse)}
\FunctionTok{library}\NormalTok{(countrycode)}
\FunctionTok{library}\NormalTok{(tmap)}
\FunctionTok{library}\NormalTok{(ggplot2)}
\end{Highlighting}
\end{Shaded}

\paragraph{Define variables}\label{define-variables}

Read csv file with global gender inequality indexes and shapefile with
world countries' spatial data.

\begin{Shaded}
\begin{Highlighting}[]
\NormalTok{world\_countries }\OtherTok{\textless{}{-}} \FunctionTok{st\_read}\NormalTok{(}\FunctionTok{here}\NormalTok{(}\StringTok{"data"}\NormalTok{,}\StringTok{"World\_Countries\_Generalized.shp"}\NormalTok{))}
\end{Highlighting}
\end{Shaded}

\begin{verbatim}
## Reading layer `World_Countries_Generalized' from data source 
##   `D:\MSC URBAN SPATIAL SCIENCE\CASA005 GIS\WEEK 4\HOMEWORK\GIS-HW-W4\data\World_Countries_Generalized.shp' 
##   using driver `ESRI Shapefile'
## Simple feature collection with 251 features and 4 fields
## Geometry type: MULTIPOLYGON
## Dimension:     XY
## Bounding box:  xmin: -20037510 ymin: -30240970 xmax: 20037510 ymax: 18418390
## Projected CRS: WGS 84 / Pseudo-Mercator
\end{verbatim}

\begin{Shaded}
\begin{Highlighting}[]
\NormalTok{inequality\_data }\OtherTok{\textless{}{-}} \FunctionTok{read.csv}\NormalTok{(}\FunctionTok{here}\NormalTok{(}\StringTok{"data"}\NormalTok{,}\StringTok{"HDR23{-}24\_Composite\_indices\_complete\_time\_series.csv"}\NormalTok{))}
\end{Highlighting}
\end{Shaded}

\paragraph{Standardize country codes}\label{standardize-country-codes}

As both variables have different type of country codes, it is necessary
to standardize them, in this case to ISO2C type. Also, as there are some
values that don't correspond to country codes, we define ``warn=FALSE'',
in order to ignore them.

\begin{Shaded}
\begin{Highlighting}[]
\NormalTok{inequality\_data}\SpecialCharTok{$}\NormalTok{country\_code }\OtherTok{\textless{}{-}} \FunctionTok{countrycode}\NormalTok{(inequality\_data}\SpecialCharTok{$}\NormalTok{iso3, }\AttributeTok{origin=}\StringTok{"iso3c"}\NormalTok{, }\AttributeTok{destination=}\StringTok{"iso2c"}\NormalTok{, }\AttributeTok{warn=}\StringTok{"FALSE"}\NormalTok{)}
\end{Highlighting}
\end{Shaded}

\paragraph{Clean and wrangle data
sets}\label{clean-and-wrangle-data-sets}

In order to manage smaller data sets, only necessary columns will be
selected using dplyr library. Also, a new column including the
difference between de gender inequality index between 2019 and 2010,
will be added.

\begin{Shaded}
\begin{Highlighting}[]
\NormalTok{inequality\_data\_clean }\OtherTok{\textless{}{-}}\NormalTok{  inequality\_data }\SpecialCharTok{\%\textgreater{}\%}
\NormalTok{  dplyr}\SpecialCharTok{::}\FunctionTok{select}\NormalTok{(}\StringTok{"country\_code"}\NormalTok{,}\StringTok{"hdi\_2010"}\NormalTok{,}\StringTok{"hdi\_2019"}\NormalTok{)}

\NormalTok{world\_countries\_clean }\OtherTok{\textless{}{-}}\NormalTok{ world\_countries }\SpecialCharTok{\%\textgreater{}\%}
\NormalTok{  dplyr}\SpecialCharTok{::}\FunctionTok{select}\NormalTok{(}\StringTok{"ISO"}\NormalTok{, }\StringTok{"COUNTRY"}\NormalTok{,}\StringTok{"geometry"}\NormalTok{)}

\NormalTok{inequality\_data\_clean }\OtherTok{\textless{}{-}}\NormalTok{ inequality\_data\_clean }\SpecialCharTok{\%\textgreater{}\%}
    \FunctionTok{mutate}\NormalTok{(}\AttributeTok{hdi\_difference=}\NormalTok{ (hdi\_2019 }\SpecialCharTok{{-}}\NormalTok{ hdi\_2010))}
\end{Highlighting}
\end{Shaded}

\paragraph{Join inequality data to spatial
data}\label{join-inequality-data-to-spatial-data}

Using a left join, both data sets will be merged by their country codes.

\begin{Shaded}
\begin{Highlighting}[]
\NormalTok{countries\_inequality }\OtherTok{\textless{}{-}}\NormalTok{ world\_countries\_clean }\SpecialCharTok{\%\textgreater{}\%}
  \FunctionTok{left\_join}\NormalTok{(.,}
\NormalTok{            inequality\_data\_clean,}
            \AttributeTok{by=}\FunctionTok{c}\NormalTok{(}\StringTok{"ISO"}\OtherTok{=}\StringTok{"country\_code"}\NormalTok{))}
\end{Highlighting}
\end{Shaded}

\paragraph{Plot data}\label{plot-data}

A quick thematic map is created. This map presents in a darker green the
countries that have increased their inequality's index the most. In a
lighter green, we can see countries that have maintained their
inequality's index, and in purple, we can see the countries that have
reduced their inequality's index in the 9 years analysed.

\begin{Shaded}
\begin{Highlighting}[]
\FunctionTok{tmap\_mode}\NormalTok{(}\StringTok{"plot"}\NormalTok{)}
\FunctionTok{qtm}\NormalTok{(countries\_inequality, }\AttributeTok{fill=} \StringTok{"hdi\_difference"}\NormalTok{)}
\end{Highlighting}
\end{Shaded}

\includegraphics{WEEK4-HOMEWORK_files/figure-latex/unnamed-chunk-6-1.pdf}

\end{document}
